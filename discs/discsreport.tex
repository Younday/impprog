\documentclass[a4paper,10pt]{article}
\usepackage{a4wide}
\usepackage[english]{babel}
\usepackage{listings}
\usepackage{color}
\usepackage{mathtools}
\definecolor{Gray}{gray}{0.95}

% This is the list style for displaying C source code:
\lstdefinestyle{code}{
    language = C,					% The language of the code snippets
    basicstyle = \small\ttfamily,	% Font of the text
    numbers = left,					% Position of the line numbers
    numberstyle = \footnotesize,	% Style of line numbers
    frame = tb,				        % Style of surrounding frame
    framextopmargin=.75mm,         % Space margin top
    framexbottommargin=.75mm,      % Space margin bottom
    framexleftmargin=2mm,          % Space margin left
    framexrightmargin=2mm,         % Space margin right
    tabsize = 3,					% Size of tab character
    breaklines = true,             % Wrap lines of code that are too long
    columns = fullflexible,			
    showstringspaces = false,
    backgroundcolor = \color{Gray}
}

% This is the list style for displaying input/output. It is different from the style above, since you don't need C keywords to be highlighted in these listings. Line numbers are also emitted in this style.
\lstdefinestyle{stdio}{
    basicstyle = \small\ttfamily,	% Font of the text
    frame = tb,				        % Style of the surrounding frame
    framextopmargin=.75mm,         % Space margin top
    framexbottommargin=.75mm,      % Space margin bottom
    framexleftmargin=2mm,          % Space margin left
    framexrightmargin=2mm,         % Space margin right
    tabsize = 3,					% Size of tab character
    breaklines = true,             % Wrap lines of text that are too long
    columns = flexible,			
    showstringspaces = false,
    backgroundcolor = \color{Gray}
}

\title{Discs assignment}
\author{Y. Moustaghfir \& S. S. Hamed\\
        S2909758 \& S2562677}
%------------------------------------------------------------%
% This is where your document starts:

\begin{document}
\maketitle

\section{Problem description}
The program needs to calculate for a number of discs in a certain area the amount of discs that do not overlap each other. The input of the program would be first the amount of discs and after that the {\tt x} and {\tt y} and {\tt r}, which is the radius of the disc. The {\tt x} and {\tt y} value are coordinates for each disc. The precise definition of overlapping is that the discs are either on top of each other, which is of course obvious, or when the discs touch each other at the edges. 

\section{Problem analysis}
We start with the input which is given in the form of {\tt x, y, r} and of course the number of discs we want to have. With this information you have to calculate each position of the discs in that area. With that information it is possible to calculate the exact positions of all the discs

\section{Design}


\section{Program code}
\begin{lstlisting}[style = code, title = market.c]
/*
 * program: discs.c
 * 
 * Copyright 2015 Younes Moustaghfir, Sharif Hamed
 */ 

#include <stdio.h>
#include <stdlib.h>
#include <math.h>

void *safeMalloc(int size) {
  void *ptr = malloc(size);
  if (ptr == NULL) {
    printf("\nError: memory allocation failed....abort\n");
    exit(-1);
  }
  return ptr;
}

int **makeDiscs(int sz) {
  int row, **arr;
  arr = safeMalloc(sz*sizeof(int *));
  for (row=0; row<sz; row++) {
    arr[row] = safeMalloc(4*sizeof(int));
  }
  return arr;
}

int intersect(int disc1, int disc2, int **discs) {
	int sxD, syD, count = 0;
	int xD = discs[disc1][0] - discs[disc2][0];
	int yD = discs[disc1][1] - discs[disc2][1];
	sxD = xD * xD;
	syD = yD * yD;
	if(sxD + syD <= (discs[disc1][2]+discs[disc2][2])*(discs[disc1][2]+discs[disc2][2])) {
		count = 1;
	}
	return count;
}

int countIntersect(int sz, int **discs) {
	int count = 0;
	for(int i = 0; i < (sz-1); i++) {
		for(int j = i + 1; j < sz; j++) {
			if(intersect(i, j, discs)) {
				discs[i][3] = 1;
				discs[j][3] = 1;
			}
		}
	}
	for(int i = 0; i < sz; i++) {
		if(discs[i][3]) {
			count++;
		}
	}
	return count;
}



int main(int argc, char **argv)
{
	int count;
	int sz;
	int **discs;
	scanf("%d\n", &sz);
	discs = makeDiscs(sz);
	for(int i = 0; i < sz; i++) {
		scanf("%d %d %d\n", &discs[i][0], &discs[i][1], &discs[i][2]);
	}
	count = countIntersect(sz, discs);
	printf("number of discs: %d\n", sz);
	printf("number of non-overlapping discs: %d\n", sz-count);
	return 0;
}

\end{lstlisting}

\section{Test results}

\begin{itemize}

\item Input: (prices of the oranges, grapefruits and melons in cents)
\begin{lstlisting}[style = stdio]
88
99
102
\end{lstlisting}

  Output:
\begin{lstlisting}[style = stdio]
1 orange, 62 grapefruits, 37 melons
4 oranges, 48 grapefruits, 48 melons
7 oranges, 34 grapefruits, 59 melons
10 oranges, 20 grapefruits, 70 melons
13 oranges, 6 grapefruits, 81 melons
\end{lstlisting}


\item Input: (prices of the oranges, grapefruits and melons in cents)
\begin{lstlisting}[style = stdio]
100
0
0
\end{lstlisting}

  Output:
\begin{lstlisting}[style = stdio]
100 oranges, 0 grapefruits, 0 melons
\end{lstlisting}


\item Input: (prices of the oranges, grapefruits and melons in cents)
\begin{lstlisting}[style = stdio]
99
100
101
\end{lstlisting}

  Output:
\begin{lstlisting}[style = stdio]
0 oranges, 100 grapefruits, 0 melons
1 orange, 98 grapefruits, 1 melon
2 oranges, 96 grapefruits, 2 melons
3 oranges, 94 grapefruits, 3 melons
4 oranges, 92 grapefruits, 4 melons
5 oranges, 90 grapefruits, 5 melons
6 oranges, 88 grapefruits, 6 melons
7 oranges, 86 grapefruits, 7 melons
8 oranges, 84 grapefruits, 8 melons
9 oranges, 82 grapefruits, 9 melons
10 oranges, 80 grapefruits, 10 melons
11 oranges, 78 grapefruits, 11 melons
12 oranges, 76 grapefruits, 12 melons
13 oranges, 74 grapefruits, 13 melons
14 oranges, 72 grapefruits, 14 melons
15 oranges, 70 grapefruits, 15 melons
16 oranges, 68 grapefruits, 16 melons
17 oranges, 66 grapefruits, 17 melons
18 oranges, 64 grapefruits, 18 melons
19 oranges, 62 grapefruits, 19 melons
20 oranges, 60 grapefruits, 20 melons
21 oranges, 58 grapefruits, 21 melons
22 oranges, 56 grapefruits, 22 melons
23 oranges, 54 grapefruits, 23 melons
24 oranges, 52 grapefruits, 24 melons
25 oranges, 50 grapefruits, 25 melons
26 oranges, 48 grapefruits, 26 melons
27 oranges, 46 grapefruits, 27 melons
28 oranges, 44 grapefruits, 28 melons
29 oranges, 42 grapefruits, 29 melons
30 oranges, 40 grapefruits, 30 melons
31 oranges, 38 grapefruits, 31 melons
32 oranges, 36 grapefruits, 32 melons
33 oranges, 34 grapefruits, 33 melons
34 oranges, 32 grapefruits, 34 melons
35 oranges, 30 grapefruits, 35 melons
36 oranges, 28 grapefruits, 36 melons
37 oranges, 26 grapefruits, 37 melons
38 oranges, 24 grapefruits, 38 melons
39 oranges, 22 grapefruits, 39 melons
40 oranges, 20 grapefruits, 40 melons
41 oranges, 18 grapefruits, 41 melons
42 oranges, 16 grapefruits, 42 melons
43 oranges, 14 grapefruits, 43 melons
44 oranges, 12 grapefruits, 44 melons
45 oranges, 10 grapefruits, 45 melons
46 oranges, 8 grapefruits, 46 melons
47 oranges, 6 grapefruits, 47 melons
48 oranges, 4 grapefruits, 48 melons
49 oranges, 2 grapefruits, 49 melons
50 oranges, 0 grapefruits, 50 melons
\end{lstlisting}

\end{itemize}

\section{Evaluation}
The program was in fact easy to design to pass simple tests like the first test result, but when we had to check for, for example the value of {\tt mbought}, the program logic of the if-statements became somewhat more complex. At first we had some trouble exactly figuring out what the exact logic should be, but with some help and a lot of thought we got the program to pass all the test cases from Justitia. 

We have chosen to write down all the possible statements that we could print instead of splitting the {\tt printf} statements. We could have checked for each part of the {\tt printf} statement if it should be a plural or singular form. 
\end{document}
