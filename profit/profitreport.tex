\documentclass[a4paper,10pt]{article}
\usepackage{a4wide}
\usepackage[english]{babel}
\usepackage{listings}
\usepackage{color}
\usepackage{mathtools}
\definecolor{Gray}{gray}{0.95}

% This is the list style for displaying C source code:
\lstdefinestyle{code}{
    language = C,					% The language of the code snippets
    basicstyle = \small\ttfamily,	% Font of the text
    numbers = left,					% Position of the line numbers
    numberstyle = \footnotesize,	% Style of line numbers
    frame = tb,				        % Style of surrounding frame
    framextopmargin=.75mm,         % Space margin top
    framexbottommargin=.75mm,      % Space margin bottom
    framexleftmargin=2mm,          % Space margin left
    framexrightmargin=2mm,         % Space margin right
    tabsize = 3,					% Size of tab character
    breaklines = true,             % Wrap lines of code that are too long
    columns = fullflexible,			
    showstringspaces = false,
    backgroundcolor = \color{Gray}
}

% This is the list style for displaying input/output. It is different from the style above, since you don't need C keywords to be highlighted in these listings. Line numbers are also emitted in this style.
\lstdefinestyle{stdio}{
    basicstyle = \small\ttfamily,	% Font of the text
    frame = tb,				        % Style of the surrounding frame
    framextopmargin=.75mm,         % Space margin top
    framexbottommargin=.75mm,      % Space margin bottom
    framexleftmargin=2mm,          % Space margin left
    framexrightmargin=2mm,         % Space margin right
    tabsize = 3,					% Size of tab character
    breaklines = true,             % Wrap lines of text that are too long
    columns = flexible,			
    showstringspaces = false,
    backgroundcolor = \color{Gray}
}

\title{Maximum profit assignment}
\author{Y. Moustaghfir \& S. S. Hamed\\
        S2909758 \& S2562677}
%------------------------------------------------------------%
% This is where your document starts:

\begin{document}
\maketitle

\section{Problem description}
 

\section{Problem analysis}



\section{Design}


\section{Program code}
\begin{lstlisting}[style = code, title = discs.c]
/*
 * profit.c
 * 
 * Copyright 2015 Younes Moustaghfir(s2909758) && Sharif Hamed (s2562677)
 * 
 */

#include <stdio.h>
#include <stdlib.h>
#include <string.h>

char *readString() {
	char longString[21];
	char *string;
	scanf("%s", longString);
	string = malloc((1+strlen(longString))*sizeof(char));
	strcpy(string, longString);
	return string;
}

void *safeMalloc(int size) {
  void *ptr = malloc(size);
  if (ptr == NULL) {
    printf("\nError: memory allocation failed....abort\n");
    exit(-1);
  }
  return ptr;
}

int **makeArray(int sz) {
  int row, **arr;
  arr = safeMalloc(sz*sizeof(int *));
  for (row=0; row<sz; row++) {
    arr[row] = safeMalloc(2*sizeof(int));
  }
  return arr;
}

int countProfit(int **array, int pr, int products) {
	int firstCase, secondCase;
	//First base case: production resources is 0 or below 0 or we've run out of products
	if(products == 0 || pr <= 0) {
		return 0;
	}
	firstCase = countProfit(array, pr, products - 1); //This is to decrement over the different products
	if(pr >= array[products - 1][0]) { //If the avaialble production resources are bigger than the cost of the product, then we can iniate the recursive function
		secondCase = countProfit(array, (pr - array[products - 1][0]), (products - 1)) + array[products - 1][1]; 
		if(secondCase > firstCase) {
			return secondCase; 
		}
	}
	return firstCase;
}

int main(int argc, char **argv)
{
	int size, pr;
	int **array;
	char *word;
	
	printf("number of products: ");
	scanf("%d", &size);

	array = makeArray(size);
	for(int i = 0; i < size; i++) {
		printf("product: ");
		word = readString();
		printf("resource cost for %s: ", word);
		scanf("%d", &array[i][0]);
		printf("profit for %s: ", word);
		scanf("%d", &array[i][1]);
	}

	printf("available production resources: ");
	scanf("%d", &pr);
	int a = countProfit(array, pr, size);
	
	printf("Maximum profit: %d\n", a);
	
	return 0;
}

\end{lstlisting}

\section{Test results}

\begin{itemize}

\item Input: First the size of the array, then coordinates and radius. ({\tt x}, {\tt y} and {\tt r})
\begin{lstlisting}[style = stdio]
5
mustard
8
20
ketchup
5
15
wine
7
19
beer
5
14
mayonnaise
4
13
14
\end{lstlisting}

  Output:
\begin{lstlisting}[style = stdio]
Maximum profit: 42
\end{lstlisting}


\item Input:
\begin{lstlisting}[style = stdio]
5
  2   4   5
  3   7   1
-2   -3   3
-12   9  10
  9   5   6
\end{lstlisting}

  Output:
\begin{lstlisting}[style = stdio]
number of discs: 5
number of non-overlapping discs: 1
\end{lstlisting}


\item Input: 
\begin{lstlisting}[style = stdio]
3
 -5   3   4
  3  -7   1
 -2  -3   3
\end{lstlisting}

  Output:
\begin{lstlisting}[style = stdio]
number of discs: 
number of non-overlapping discs: 1
\end{lstlisting}

\end{itemize}

\section{Evaluation}

\end{document}
