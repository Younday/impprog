\documentclass[a4paper,10pt]{article}
\usepackage{a4wide}
\usepackage[english]{babel}
\usepackage{listings}
\usepackage{color}
\usepackage{mathtools}
\definecolor{Gray}{gray}{0.95}

% This is the list style for displaying C source code:
\lstdefinestyle{code}{
    language = C,					% The language of the code snippets
    basicstyle = \small\ttfamily,	% Font of the text
    numbers = left,					% Position of the line numbers
    numberstyle = \footnotesize,	% Style of line numbers
    frame = tb,				        % Style of surrounding frame
    framextopmargin=.75mm,         % Space margin top
    framexbottommargin=.75mm,      % Space margin bottom
    framexleftmargin=2mm,          % Space margin left
    framexrightmargin=2mm,         % Space margin right
    tabsize = 3,					% Size of tab character
    breaklines = true,             % Wrap lines of code that are too long
    columns = fullflexible,			
    showstringspaces = false,
    backgroundcolor = \color{Gray}
}

% This is the list style for displaying input/output. It is different from the style above, since you don't need C keywords to be highlighted in these listings. Line numbers are also emitted in this style.
\lstdefinestyle{stdio}{
    basicstyle = \small\ttfamily,	% Font of the text
    frame = tb,				        % Style of the surrounding frame
    framextopmargin=.75mm,         % Space margin top
    framexbottommargin=.75mm,      % Space margin bottom
    framexleftmargin=2mm,          % Space margin left
    framexrightmargin=2mm,         % Space margin right
    tabsize = 3,					% Size of tab character
    breaklines = true,             % Wrap lines of text that are too long
    columns = flexible,			
    showstringspaces = false,
    backgroundcolor = \color{Gray}
}

\title{Market assignment}
\author{Y. Moustaghfir \& S. S. Hamed\\
        S2909758 \& S2562677}
%------------------------------------------------------------%
% This is where your document starts:

\begin{document}
\maketitle

\section{Problem description}
The program is required to print all the possible options of buying exactly 100 pieces of fruit with exactly 100 euros. We can choose between oranges, grapefruits and melons. The input of the program is the price of each fruit(orange, grapefruit and melons) in eurocents. 

\section{Problem analysis}
We interpret the problem as follows: we first have to loop over the first kind of fruit. This means that we have to loop for example to 100 pieces of oranges. Within that first loop(let's call it the orange loop), we have to loop over the second kind of fruit(the grapefruit loop). After that we would have to make sure that the leftover fruit is set to melons, so that we can analyze the amount and price of that fruit as well.
This interpretation is motivated by the following statements:

\section{Design}
First the prices for the oranges, grapefruits and melons are read from the standard input and stored (in the variables {\tt op}, {\tt gp} and {\tt mp}). 
After that we begin with the first for loop with the variable {\tt obought} which stands for the amount of oranges. Within the orange loop, we start another for loop with the variable {\tt gbought}, which stands for the amount of grapefruit. At last we calculate the amount of melons, which is: $$100 - obought - gbought$$We store that value in the variable {\tt mbought}.
After we stored all the values of the amount of fruit, we begin to calculate the total amount of fruit and store that value in the variable {\\tt total}. We also need the total price of the fruit to calculate if we have spent exactly 100 euros. This value is stored in the variable {\tt totalp} with the expression: $$totalp = (gbought * gp) + (obought * op) + (mbought * mp)$$

After all this looping, we start to select the statements we want to print. Each if statement checks if the price is exactly 10000 cents(because our input is in cents) and the total amount of fruit is 100. Each if statement prints a certain plural or singular form of the noun of the fruit. The if-statement check all the possibilities and print out the correct form of the noun(singular or plural). The oranges and grapefruit are unable to reach a value under 0, but {\\tt mbought} is able to reach a value under 0. That is why a lot of if-statement checks also if the value of {\tt mbought} is not smaller than 0.

\section{Program code}
\begin{lstlisting}[style = code, title = market.c]
/* File : market.c
 * Author: Younes Moustaghfir
 */

#include <stdio.h>
#include <stdlib.h>

int main(int argc, char *argv[]) {
	int obought, gbought, mbought, op, gp, mp, total, totalp;
	
	printf("price of an orange: ");
	scanf("%d", &op);
	printf("price of a grapefruit: ");
	scanf("%d", &gp);
	printf("price of a melon: ");
	scanf("%d", &mp);
	
	for(obought = 0; obought <= 100; obought++) {
		for(gbought = 0; gbought <= 100; gbought++) {
			mbought = 100 - obought - gbought;
			total = obought + gbought + mbought;
			totalp = (gbought*gp) + (obought*op) + (mbought*mp);
			if(totalp == 10000 && total == 100 && obought != 1 && gbought != 1 && mbought != 1 && !(mbought < 0)) {
				printf("%d oranges, ", obought);
				printf("%d grapefruits, ", gbought);
				printf("%d melons\n", mbought);
			}
			else if(totalp == 10000 && total == 100 && obought == 1 && mbought != 1 && gbought != 1 && !(mbought < 0)) {
				printf("%d orange, %d grapefruits, %d melons\n", obought, gbought, mbought); 
			}
			else if(totalp == 10000 && total == 100 && mbought == 1 && gbought != 1 && obought != 1) {
				printf("%d oranges, %d grapefruits, %d melon\n", obought, gbought, mbought);
			}
			else if(totalp == 10000 && total == 100 && gbought == 1 && mbought != 1 && obought != 1 && !(mbought < 0)) {
				printf("%d oranges, %d grapefruit, %d melons\n", obought, gbought, mbought);
			}
			else if(totalp == 10000 && total == 100 && gbought == 1 && obought == 1 && mbought != 1 && !(mbought <0)) {
				printf("%d orange, %d grapefruit, %d melons\n", obought, gbought, mbought);
			}
			else if(totalp == 10000 && total == 100 && gbought == 1 && mbought == 1 && obought != 1) {
				printf("%d oranges, %d grapefruit, %d melon\n", obought, gbought, mbought);
			}
			else if(totalp == 10000 && total == 100 && obought == 1 && mbought == 1 && gbought != 1) {
				printf("%d orange, %d grapefruits, %d melon\n", obought, gbought, mbought);
			}
			else if(totalp == 10000 && total == 100 && obought == 1 && gbought == 1 && mbought == 1) {
				printf("%d orange, %d grapefruit, %d melon\n", obought, gbought, mbought);
			}
		}
	}
}
\end{lstlisting}

\section{Test results}

\begin{itemize}

\item Input: (prices of the oranges, grapefruits and melons in cents)
\begin{lstlisting}[style = stdio]
88
99
102
\end{lstlisting}

  Output:
\begin{lstlisting}[style = stdio]
1 orange, 62 grapefruits, 37 melons
4 oranges, 48 grapefruits, 48 melons
7 oranges, 34 grapefruits, 59 melons
10 oranges, 20 grapefruits, 70 melons
13 oranges, 6 grapefruits, 81 melons
\end{lstlisting}


\item Input: (prices of the oranges, grapefruits and melons in cents)
\begin{lstlisting}[style = stdio]
100
0
0
\end{lstlisting}

  Output:
\begin{lstlisting}[style = stdio]
100 oranges, 0 grapefruits, 0 melons
\end{lstlisting}


\item Input: (prices of the oranges, grapefruits and melons in cents)
\begin{lstlisting}[style = stdio]
99
100
101
\end{lstlisting}

  Output:
\begin{lstlisting}[style = stdio]
0 oranges, 100 grapefruits, 0 melons
1 orange, 98 grapefruits, 1 melon
2 oranges, 96 grapefruits, 2 melons
3 oranges, 94 grapefruits, 3 melons
4 oranges, 92 grapefruits, 4 melons
5 oranges, 90 grapefruits, 5 melons
6 oranges, 88 grapefruits, 6 melons
7 oranges, 86 grapefruits, 7 melons
8 oranges, 84 grapefruits, 8 melons
9 oranges, 82 grapefruits, 9 melons
10 oranges, 80 grapefruits, 10 melons
11 oranges, 78 grapefruits, 11 melons
12 oranges, 76 grapefruits, 12 melons
13 oranges, 74 grapefruits, 13 melons
14 oranges, 72 grapefruits, 14 melons
15 oranges, 70 grapefruits, 15 melons
16 oranges, 68 grapefruits, 16 melons
17 oranges, 66 grapefruits, 17 melons
18 oranges, 64 grapefruits, 18 melons
19 oranges, 62 grapefruits, 19 melons
20 oranges, 60 grapefruits, 20 melons
21 oranges, 58 grapefruits, 21 melons
22 oranges, 56 grapefruits, 22 melons
23 oranges, 54 grapefruits, 23 melons
24 oranges, 52 grapefruits, 24 melons
25 oranges, 50 grapefruits, 25 melons
26 oranges, 48 grapefruits, 26 melons
27 oranges, 46 grapefruits, 27 melons
28 oranges, 44 grapefruits, 28 melons
29 oranges, 42 grapefruits, 29 melons
30 oranges, 40 grapefruits, 30 melons
31 oranges, 38 grapefruits, 31 melons
32 oranges, 36 grapefruits, 32 melons
33 oranges, 34 grapefruits, 33 melons
34 oranges, 32 grapefruits, 34 melons
35 oranges, 30 grapefruits, 35 melons
36 oranges, 28 grapefruits, 36 melons
37 oranges, 26 grapefruits, 37 melons
38 oranges, 24 grapefruits, 38 melons
39 oranges, 22 grapefruits, 39 melons
40 oranges, 20 grapefruits, 40 melons
41 oranges, 18 grapefruits, 41 melons
42 oranges, 16 grapefruits, 42 melons
43 oranges, 14 grapefruits, 43 melons
44 oranges, 12 grapefruits, 44 melons
45 oranges, 10 grapefruits, 45 melons
46 oranges, 8 grapefruits, 46 melons
47 oranges, 6 grapefruits, 47 melons
48 oranges, 4 grapefruits, 48 melons
49 oranges, 2 grapefruits, 49 melons
50 oranges, 0 grapefruits, 50 melons
\end{lstlisting}

\end{itemize}

\section{Evaluation}
The program was in fact easy to design to pass simple tests like the first test result, but when we had to check for, for example the value of {\tt mbought}, the program logic of the if-statements became somewhat more complex. At first we had some trouble exactly figuring out what the exact logic should be, but with some help and a lot of thought we got the program to pass all the test cases from Justitia. 

We have chosen to write down all the possible statements that we could print instead of splitting the {\tt printf} statements. We could have checked for each part of the {\tt printf} statement if it should be a plural or singular form. 
\end{document}
